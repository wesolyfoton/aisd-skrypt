\section{Problem plecakowy}

\label{sec:plecaki}

W tym rozdziale opiszemy jeden z najbardziej znanych problemów optymalizacyjnych, czyli tak zwany problem plecakowy.

Załóżmy, że jesteśmy złodziejem, który w trakcie włamania chce wypełnić swój plecak wielkości $W$ przedmiotami o jak największej wartości.
Mamy $M$ typów przedmiotów i plecak wielkości $W$.
Każdy z przedmiotów typu $i$ ma swoją wartość $v_i> 0$ i wielkość $w_i> 0$.
To ile przedmiotów typu $i$ zabierzemy będziemy oznaczać przez $x_i\epsilon N$.
Dyskretny plecakowy posiada trzy popularne wersje.


\begin{itemize}
  \item 0-1 dyskretny problem plecakowy, w którym $x_i \epsilon \left \{ 0, 1 \right \}$.
  Tę wersję możemy interpretować, jako okradanie galerii sztuki.
  \item Ograniczony problem placakowy, w którym, dla każdego $i$ mamy podaną ilość przedmiotów danego typu - $c_i$, czyli $x_i \leq c_i$.
  Tę wersję możemy interpretować, jak zwykłą kradzież ze sklepu rtv.
  \item Nieograniczony problem plecakowy, w którym $x_i$ nie ma górnego ograniczenia.
  Tę wersję możemy intepretować, jako kradzież kopii Windowsa, gdzie plecakiem będzie dysk.
\end{itemize}

\subsection{Ciągły problem plecakowy}
Problem plecakowy posiada także wersję ciągłą, w której możemy zabrać część, zamiast całości przedmiotu.
Specyfikacja jest analogiczna do wariantu dyskretnego, z tą różnicą, że $x_i \epsilon R_+ \cup \left \{ 0 \right \}$.
O tym wariancie możemy myśleć jak o kradzieży płynnych chemikaliów.

\subsection{Próba rozwiązania zachłannego}
Mimo, że intuicyjnym byłoby dla każdego przedmiotu wyliczyć stosunek wartości do wielkości i branie przedmiotów
o najlepszym współczynniku rozwiązanie takie jest nieprawidłowe.
Prostym kontrprzykładem będzie sytuacja, w której plecak ma wielkość $w = 10$ i do dyspozycji mamy $3$ przedmioty;
$v_1=9$, $v_2=v_3=5$, $w_1=6$, $w_2=w_3=5$. W każdym z wariantów problemu postępując w ten sposób zachłannie wybierzemy przedmiot $1$,
otrzymując plecak o wartości $9$, zamiast wybierając $2$ i $3$ otrzymując plecak o wartości $10$.
Co ciekawe podejście zachłanne sprawdza się w przypadku problemu ciągłego.

\subsection{Rozwiązanie dynamiczne}
\subsubsection{0-1}
Stwórzmy dwuwymiarową tablicę A.
$A[i, w]$ będzie oznaczała maksymalną wartość plecaku wielkości $w$, rozważając $i$ pierwszych przedmiotów.
Uznajmy przy tym, że dla indeksów ujemnych A będzie zwracać $-\infty$.

Stwórzmy zależność rekurencyjną.
\[A[0, w] = 0\]
Gdyż biorąc zero przedmiotów nie otrzymamy żadnej wartości dodanej.
\[A[i, 0] = 0\]
\[A[i, w] = max(A[i - 1, w], A[i - 1, w - w_i] + v_i)\]
Czyli rozważając dodanie następnego przedmiotu sprawdzamy, czy najbardziej wartościowy po jego dołożeniu jest wart więcej, niż najbardziej wartościowy plecak bez tego elementu.

Wynik będzie w komórce $A[M, W]$.

\subsubsection{Ograniczony}
\[A[0, w] = 0\]
\[A[i, 0] = 0\]
\[A[i, w] = max(\left \{ A[i - 1, w - w_i \cdot n] + v_i \cdot n ~|~ 0 \leq n \leq c_i \right \})\]
Podobnie, do wariantu 0-1, z tym, że tutaj przy i-tym elemencie rozważamy jego każdą możliwą ilość.

Wynik będzie w komórce $A[M, W]$.

\subsubsection{Nieograniczony}
W wariance nieograniczonym; jako, że nie musimy kontrolować, ile elementów wzięliśmy, sytuacja się upraszcza.
Wystarczy nam jednowymiarowa tablica A; A[w] będzie oznaczała wartość najcennieszego plecaka o wielkości w.
\[A[0] = 0\]
\[A[w] = max(\left \{ A[w - w_i] + c_i ~|~ w_i <= w \right \})\]

Wynik będzie w komórce $A[W]$.

\subsubsection{Złożoność}
\subsubsubsection{0-1}
Złożoność pseudowielomioanowa $O(M \cdot W)$.
\subsubsection{Ograniczony}
Złożoność pseudowielomioanowa $O(M \cdot W)$.
\subsubsection{Nieograniczony}
Złożoność pseudowielomioanowa $O(M \cdot W \cdot max(\left \{ c_i \right \}))$.
\comment{Wydaje mi się, że na wykładzie było użyte O, a nie $\Theta$, więc tutaj zrobiłem tak samo}
