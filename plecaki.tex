\section{Problem plecakowy}

\label{sec:plecaki}

W tym rozdziale opiszemy jeden z najbardziej znanych problemów optymalizacyjnych, czyli tak zwany problem plecakowy.

Załóżmy, że jesteśmy złodziejem, który w trakcie włamania chce wypełnić swój plecak wielkości $W$ przedmiotamiple o jak największej wartości.
Każdy z przedmiotów $i$ ma swoją wartość $v_i$ i wielkość $w_i$.
To ile przedmiotów typu $i$ zabierzemy będziemy oznaczać przez $x_i \geq 0$.
Problem plecakowy posiada aż trzy(!!!) popularne warianty.


\begin{itemize}
  \item 0-1 problem plecakowy, w którym $x_i \epsilon \left \{ 0, 1 \right \}$.
  Tę wersję możemy interpretować jako okradanie gelerii sztuki.
  \item Ograniczony problem placakowy, w którym, dla każdego $i$ mamy podaną ilość przedmiotów danego typu - $c_i$, czyli $x_i \leq c_i$.
  Tę wersję możemy interpretować jak zwykłą kradzież ze sklepu rtv.
  \item Nieograniczony problem plecakowy, w którym $x_i$ nie ma górnego ograniczenia.
  Tę wersję można interpretować jako kradzież ziaren piasku z Sahary.
\end{itemize}

\subsection{Próba rozwiązania zachłannego}
Mimo, że intuicyjnym byłoby dla każdego przedmiotu wyliczyć stosunek wartości do wielkości i branie przedmiotów
o najlepszym współczynniku rozwiązanie takie jest nieprawidłowe.
Prostym kontrprzykładem będzie sytuacja, w której plecak ma wielkość $w = 10$ i do dyspozycji mamy 3 przedmioty;
$v_1=9$, $v_2=v_3=5$, $w_1=6$, $w_2=w_3=5$. W każdym z wariantów problemu postępując w ten sposób zachłannie wybierzemy przedmiot 1,
otrzymując plecak o wartości 9, zamiast wybierając 2 i 3 otrzymując plecak o wartości 10.
Co ciekawe; przy założeniu, że możemy brać część, a nie całość przedmiotu podejście zachłanne jest poprawne.

\subsection{Rozwiązanie dynamiczne}
\subsubsection{0-1}
Rozwiążmy na początku problem dla pierwszego wariantu(pozostałe będą bardzo podobne).
Stwórzmy dwuwymiarową tablicę A.
$A[i, w]$ będzie oznaczała maksymalną wartość plecaku wielkości $w$, rozważając $i$ pierwszych przedmiotów.
Uznajmy przy tym, że dla indeksów ujemnych A będzie zwracać $-\infty$.

Stwórzmy zależność rekurencyjną.
\[A[0, w] = 0\]
Gdyż biorąc zero przedmiotów nie otrzymamy żadnej wartości dodanej.
\[A[i, 0] = 0\]
\[A[i, w] = max(m[i - 1, w], m[i - 1, w - w_i] + v_i)\]
Czyli rozważając dodanie następnego przedmiotu sprawdzamy, czy najbardziej wartościowy po jego dołożeniu jest wart więcej, niż najbardziej wartościowy plecak bez tego elementu.

\subsubsection{Ograniczony}
\[A[0, w] = 0\]
\[A[i, 0] = 0\]
\[A[i, w] = max(\left \{ m[i - 1, w - w_i * n] + v_i * n ~|~ 0 \leq n \leq c_i \wedge w - w_i * n \geq 0 \right \})\]
Podobnie, do wariantu 0-1, z tym, że tutaj przy i-tym elemencie rozważamy jego każdą możliwą ilość.
Wariant ograniczony możemy też łatwo stransformować na wariant 0-1.

\subsubsection{Nieograniczony}
W wariance nieograniczonym; jako, że nie musimy kontrolować, ile elementów wzięliśmy, sytuacja się upraszcza.
Wystarczy nam jednowymiarowa tablica A; A[w] będzie oznaczała wartość najcennieszego plecaka o wielkości w.
\[A[0] = 0\]
\[A[w] = max(\left \{ A[w - w_i] + c_i ~|~ w_i <= w \right \})\]

\subsubsection{Złożoność}
Każdy z podanych algorytmów ma złożoność pseudowielomioanową $O(W * n)$.
\comment{Wydaje mi się, że na wykładzie było użyte O, a nie $\Theta$, więc tutaj zrobiłem tak samo}
