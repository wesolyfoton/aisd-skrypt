\section{Algorytmy sortowania}

W tym rozdziale zapoznamy się z algorytmem sortowania przez scalanie (ang. \textit{merge sort}). 
Wykorzystuje on metodę "dziel i zwyciężaj" - problem jest dzielony na kilka mniejszych podproblemów podobnych do początkowego problemu, problemy te są rozwiązywane rekurencyjnie, a następnie rozwiązania otrzymane dla podproblemów scala się, uzyskując rozwiązanie całego zadania.

\textbf{Idea.} Algorytm sortujący dzieli porządkowany $n$-elementowy zbiór na kolejne połowy, aż do uzyskania $n$ jednoelementowych zbiorów - każdy taki zbiór jest już posortowany. 
Uzyskane w ten sposób części zbioru sortuje rekurencyjnie - posortowane części łączy ze sobą za pomocą scalania tak, aby wynikowy zbiór był posortowany.

\textbf{Scalanie.} Podstawową operacją algorytmu jest scalanie dwóch uporządkowanych zbiorów w jeden uporządkowany zbiór.
W celu wykonania scalania skorzystamy z pomocniczej procedury  \texttt{merge}$(A,p,q,r)$, gdzie $A$ jest tablicą, a $p, q, r$ są indeksami takimi, że $p \leq q < r$. 
W procedurze zakłada się, że tablice $A[p..q]$ oraz $A[q+1..r]$ (dwie przyległe połówki zbioru, który został przez ten algorytm podzielony) są posortowane. 
Procedura \texttt{merge} scala te tablice w jedną posortowaną tablicę $A[p..r]$. 
Ogólna zasada działania jest następująca:
\begin{enumerate}
\item Przygotuj pusty zbiór tymczasowy.
\item Dopóki żaden ze scalanych zbiorów nie wyczerpał elementów, porównuj ze sobą pierwsze elementy każdego z nich i w zbiorze tymczasowym umieszczaj mniejszy z elementów.
\item W zbiorze tymczasowym umieść zawartość tego scalanego zbioru, który zawiera niewykorzystane jeszcze elementy.
\item Zawartość zbioru tymczasowego przepisz do zbioru wynikowego i zakończ algorytm.
\end{enumerate}
Zapis algorytmu scalania dwóch list w pseudokodzie podano niżej.
\begin{algorithm}[h]

  \DontPrintSemicolon
  
  \SetAlgorithmName{Algorytm}{}
  
  \KwData{ tablica $A$, liczby $p$, $q$, $r$}

  \KwResult{ posortowana tablica $A[p..r]$ }
  
  $C \leftarrow$ pusta tablica\;
  
  $i \leftarrow p$, $j \leftarrow q+1$, $k \leftarrow 0$\;
  
  \While{$i \leq q$ oraz $j \leq r$} {
  	\eIf{$A[i] \leq A[j]$} {
    	$C[k] \leftarrow A[i]$, $i \leftarrow i+1$\;
    } { 
    	$C[k] \leftarrow A[j]$, $j \leftarrow j+1$\;
    
    }
    $k \leftarrow k+1$\;
  
  }

  \While{$i \leq q$} {
    $C[k] \leftarrow A[i]$, $i \leftarrow i+1$, $k \leftarrow k+1$\;
  }
  
  \While{$j \leq r$} {
    $C[k] \leftarrow A[j]$, $j \leftarrow j+1$, $k \leftarrow k+1$\;
  }
  
  \caption{Procedura \texttt{merge}}
  \label{alg-merge}
\end{algorithm}

Scalanie wymaga $O(n+m)$ operacji porównań elementów i wstawienia ich do tablicy wynikowej. 

\textbf{Sortowanie.} Algorytm sortowania przez scalanie jest algorytmem rekurencyjnym. 
Wywołuje się go z zadanymi wartościami indeksów wskazujących na początek i koniec sortowanego zbioru, zatem początkowo indeksy obejmują cały zbiór. 
Algorytm wyznacza indeks elementu połowiącego przedział, a następnie sprawdza, czy połówki zbioru zawierają więcej niż jeden element. 
Jeśli tak, to rekurencyjnie sortuje je tym samym algorytmem.
Po posortowaniu obu połówek zbioru scalamy je za pomocą opisanej wcześniej procedury scalania podzbiorów uporządkowanych i kończymy algorytm. 
Zbiór jest posortowany.
\begin{algorithm}[h]

  \DontPrintSemicolon
  
  \SetAlgorithmName{Algorytm}{}
  
  \KwData{ tablica $A$, liczby $p$, $r$}

  \KwResult{ posortowana tablica $A[p..r]$ }
  
  $q \leftarrow 0$\;
  
  \If{$p < r$} {
  	$q \leftarrow \left \lfloor{\frac{p+r}{2}}\right \rfloor $\;
    \texttt{merge sort}$(A,p,q)$\;
 	\texttt{merge sort}$(A,q+1,r)$\;
    \texttt{merge}$(A,p,q,r)$\;
  }
  \caption{Procedura \texttt{merge sort}}
  \label{alg-merge-sort}
\end{algorithm}

\textbf{Złożoność.} Chociaż algorytm sortowania przez scalanie działa poprawnie nawet wówczas, gdy $n$ jest nieparzyste, dla uproszczenia analizy załóżmy, że $n$ jest potęgą dwójki. 
Dzielimy wtedy problem na podproblemy rozmiaru dokładnie $\frac{n}{2}$. 
Rekurencję określającą czas $T(n)$ sortowania przez scalanie otrzymujemy, jak następuje. \\\\
Sortowanie przez scalanie jednego elementu wykonuje się w czasie stałym. Jeśli $n > 1$, to czas działania zależy od trzech etapów:\\
\textbf{Dziel:} podczas tego etapu znajdujemy środek przedziału, co zajmuje czas stały, zatem $D(n) = \theta(1)$.\\
\textbf{Zwyciężaj:} rozwiązujemy rekurencyjnie dwa podproblemy, każdy rozmiaru $\frac{n}{2}$, co daje czas działania $2T(\frac{n}{2})$.\\
\textbf{Połącz:} procedura \texttt{merge}, jak wspomniano wyżej, działa w czasie liniowym, a więc $P(n) = \theta(n)$.\\\\
Funkcje $D(n)$ i $P(n)$ dają po zsumowaniu funkcję rzędu $\theta(n)$. 
Dodając do tego $2T(\frac{n}{2})$ z etapu "zwyciężaj", otrzymujemy następującą rekurencję dla $T(n)$:
$$
 T(n) = 
  \begin{cases} 
   \theta(1) & n = 1 \\
   2T(\frac{n}{2}) + \theta(n) & n > 1
  \end{cases}
$$

Poniższy przykład ilustruje zasadę działania sortowania przez scalanie:\\
{\tiny<tu obrazek, ale nie umiem w obrazki, na dniach ogarnę>}

\textbf{Podsumowanie.} Sortowanie przez scalanie należy do algorytmów szybkich, posiada klasę złożoności równą $\theta(n\log n)$. Jest oparty na metodzie dziel i zwyciężaj, która powoduje podział dużego problemu na mniejsze, łatwo rozwiązywane podproblemy. Sortowanie nie odbywa się w miejscu, potrzebujemy dodatkowej struktury. Algorytm jest stabilny.

\subsection{Quick sort}

In progress
