\documentclass[10pt,b5paper]{book}
\usepackage{polski}
\usepackage{amsthm}
\usepackage[T1]{fontenc}
\usepackage[utf8]{inputenc}
\usepackage{geometry}
\usepackage{tikz}
\usepackage{amsmath}
\usepackage{amssymb}
\usepackage{array}
\usepackage[titletoc]{appendix}
\usepackage{scrextend}
\usepackage[ruled,commentsnumbered]{algorithm2e}
\usepackage{adjustbox}
\usepackage[hidelinks]{hyperref}

%% Biblioteki TiKZa

\usetikzlibrary{shapes.arrows}

%% Użyteczne komendy

\newtheorem{theorem}{Twierdzenie}
\newtheorem{definition}{Definicja}
\newtheorem{lemma}{Lemat}
\newtheorem{fact}{Fakt}
\newtheorem{observation}{Obserwacja}
\def\checkmark{\tikz\fill[scale=0.4](0,.35) -- (.25,0) -- (1,.7) -- (.25,.15) -- cycle;} 
\newcommand{\tizkboxwithcaption}[2]{\noindent\begin{figure}[!h]\centering\begin{adjustbox}{min width=0.75\textwidth, max width=1\textwidth}\input{#1}\end{adjustbox}\caption{#2}\end{figure}}

\newcommand{\comment}[1]{\marginpar{\tiny\raggedright #1}}

\renewcommand{\KwData}{\textbf{Input:}}
\renewcommand{\KwResult}{\textbf{Output:}}

%% Strona tytułowa

\usepackage[load-headings]{exsheets}
\DeclareInstance{exsheets-heading}{mylist}{default}{
  runin = true ,
  attach = {
    main[l,vc]number[l,vc](-3em,0pt) ;
    main[r,vc]points[l,vc](\linewidth+\marginparsep,0pt)
  }
}

\SetupExSheets{
  headings = mylist ,
  headings-format = \normalfont ,
  counter-format = se.qu ,
  counter-within = section
}

\usepackage{etoolbox}
\AtBeginEnvironment{question}{\addmargin[3em]{0em}}
\AtEndEnvironment{question}{\endaddmargin}

\newcommand*{\titleGM}{\begingroup
\hbox{
\hspace*{0.2\textwidth}
\hspace*{0.05\textwidth}
\parbox[b]{0.75\textwidth}{
{\noindent\Huge\bfseries Skrypt \\[0.2\baselineskip] z Algorytmów \\[0.2\baselineskip] i~struktur danych}\\[2\baselineskip]
{\large \textit{Zbiór mniej lub bardziej ciekawych algorytmów i~struktur danych, jakie bywały omawiane na wykładzie (albo i nie).}}\\[3\baselineskip]
{\Large \textsc{praca zbiorowa pod redakcją \\[0.1\baselineskip] Krzysztofa Piecucha}}

\vspace{0.5\textheight}
{\noindent Korzystać na własną odpowiedzialność.}\\[\baselineskip]
}}
\endgroup}

\definecolor{titlepagecolor}{cmyk}{0.9,1,.6,.40}

\newcommand\titlepagedecoration{%
\begin{tikzpicture}[remember picture,overlay,shorten >= -10pt]

\coordinate (aux1) at ([yshift=-15pt]current page.north west);
\coordinate (aux2) at ([yshift=-450pt]current page.north west);
\coordinate (aux3) at ([xshift=+4.5cm]current page.north west);
\coordinate (aux4) at ([yshift=-150pt]current page.north west);

\begin{scope}[titlepagecolor!60,line width=14pt,rounded corners=12pt]
\draw
  (aux1) -- coordinate (a)
  ++(-45:5) --
  ++(225:5.1) coordinate (b);
\draw[shorten <= -10pt]
  (aux3) --
  (a) --
  (aux1);
\draw[opacity=0.4,titlepagecolor,shorten <= -10pt]
  (b) --
  ++(-45:2.2) --
  ++(225:2.2);
\end{scope}
\draw[titlepagecolor,line width=9pt,rounded corners=8pt,shorten <= -10pt]
  (aux4) --
  ++(-45:1.2) --
  ++(225:1.2);
\begin{scope}[titlepagecolor!70,line width=6pt,rounded corners=6pt]
\draw[shorten <= -10pt]
  (aux2) --
  ++(-45:3) coordinate[pos=0.45] (c) --
  ++(225:3.1);
\draw
  (aux2) --
  (c) --
  ++(45:2.5) --
  ++(135:2.5) --
  ++(225:2.5) coordinate[pos=0.3] (d);   
\draw 
  (d) -- +(135:1);
\end{scope}
\end{tikzpicture}%
}

\begin{document}
\pagenumbering{arabic}
\pagestyle{empty}

\titleGM
\titlepagedecoration

\tableofcontents 
\pagestyle{plain}

\chapter{Zrobione}

\include{mastertheorem}

\include{bitoniczne}

\section{Algorytm macierzowy wyznaczania liczb Fibonacciego}

W tym rozdziale opiszemy algorytm obliczania liczb Fibonacciego, który wykorzystuje szybkie 
potęgowanie\footnote{\url{https://en.wikipedia.org/wiki/Exponentiation_by_squaring}}. 
Algorytm działa w czasie $O(\log{n})$, co sprawia, że jest znacznie atrakcyjniejszy od algorytmu 
dynamicznego, który wymaga czasu $O(n)$.

Znajdźmy taką macierz $M$, która po wymnożeniu przez transponowany wektor wyrazów 
$F_{n}$ i $F_{n - 1}$ da nam wektor, w którym otrzymamy wyrazy $F_{n + 1}$ oraz $F_{n}$. 
Łatwo sprawdzić, że dla ciągu Fibonacciego taka macierz ma postać:


\begin{equation}
	M = \begin{bmatrix}1 & 1\\1 & 0\end{bmatrix}
\end{equation}
Bo:
\begin{equation}
\label{eq:fibonacci_m}
	M \times
	\begin{bmatrix}F_n \\ F_{n - 1}\end{bmatrix}
	= \begin{bmatrix}F_{n + 1} \\ F_{n}\end{bmatrix}
\end{equation}

Wynika to wprost z definicji mnożenia macierzy oraz definicji ciągu Fibonacciego.


\begin{observation}{Zauważmy, że możemy $M$ przemnożyć przez macierz otrzymaną w \ref{eq:fibonacci_m}. Otrzymamy wtedy macierz postaci:}
\begin{equation}
	M \times (M \times \begin{bmatrix}F_n \\ F_{n - 1}\end{bmatrix})
\end{equation}
\end{observation}

\begin{observation}{A gdy zrobimy to $n$ razy...}
\label{obs:mult-n-times}
\begin{equation}
	M \times (M \times (M \times ...\, (M \times \begin{bmatrix}F_n \\ F_{n - 1}\end{bmatrix})\,...\,))
\end{equation}
\end{observation}

\begin{fact}{Mnożenie macierzy jest łączne.}
\label{fact:mult-is-associative}
\end{fact}

Z Faktu \ref{fact:mult-is-associative}. i Obserwacji \ref{obs:mult-n-times}. mamy:

\begin{equation}
	M^n \times \begin{bmatrix}F_n \\ F_{n - 1}\end{bmatrix}
\end{equation}

Pokażemy, że powyższa macierz ma zastosowanie w obliczaniu n-tej liczby Fibonacciego.

\begin{lemma}
\begin{equation}
	M^{n} \times \begin{bmatrix}F_1 \\ F_0\end{bmatrix} = \begin{bmatrix}F_{n + 1} \\ F_{n}\end{bmatrix}
\end{equation}
\end{lemma}

\begin{proof}{Przez indukcję.}\\
Sprawdźmy dla $n = 1$. Mamy:
\begin{equation}
	\begin{bmatrix}1 & 1\\1 & 0\end{bmatrix}^1 \times \begin{bmatrix}1 \\ 0\end{bmatrix}
	= \begin{bmatrix}1 \\ 1\end{bmatrix} = \begin{bmatrix}F_2 \\ F_1\end{bmatrix}
\end{equation}

Rozważmy $n + 1$ zakładając poprawność dla $n$.

\begin{equation}
	\begin{bmatrix}1 & 1\\1 & 0\end{bmatrix}^{n + 1} \times \begin{bmatrix}1 \\ 0\end{bmatrix}
	= \begin{bmatrix}1 & 1\\1 & 0\end{bmatrix} \times \begin{bmatrix}1 & 1\\1 & 0\end{bmatrix}^{n} \times \begin{bmatrix}1 \\ 0\end{bmatrix}
	= \begin{bmatrix}1 & 1\\1 & 0\end{bmatrix} \times \begin{bmatrix}F_{n+1} \\ F_{n}\end{bmatrix}
	\stackrel{\ref{eq:fibonacci_m}}{=} \begin{bmatrix}F_{n + 2} \\ F_{n + 1}\end{bmatrix}
\end{equation}
\end{proof}

\begin{algorithm}[h]
	\DontPrintSemicolon
	\SetAlgorithmName{Algorytm}{}
	
	\KwData{ n }
	
	\KwResult{ $n+1$-sza liczba Fibonacciego }

	$M \leftarrow \begin{bmatrix}1 & 1\\1 & 0\end{bmatrix}$\;
	$M' \leftarrow \texttt{exp\_by\_squaring(M, n)}$\;
	
	\KwRet{$\texttt{pierwszy element wektora}\,(M' \times \begin{bmatrix}1 \\ 0\end{bmatrix})$}\;

	\caption{Procedura \texttt{get\_fibonacci}}
\end{algorithm}

Mimo że powyższy algorytm działa w czasie $O(\log{n})$, warto mieć na uwadze fakt, że liczby Fibonacciego 
rosną wykładniczo. W praktyce oznacza to pracę na liczbach przekraczających długość słowa maszynowego.

Zaprezentowaną metodę można uogólnić na dowolne ciągi, które zdefiniowane są przez liniową 
kombinację skończonej liczby poprzednich elementów. Wystarczy znaleźć odpowiednią macierz $M$; 
dla ciągów postaci $G_{n + 1} = a_n G_n + a_{n - 1} G_{n - 1} + ... + a_{n - k} G_{n - k}$ jest to:
\begin{equation}
	M = \begin{bmatrix}a_n    & a_{n - 1} & a_{n - 2} & \dots & a_{n - k} & a_{n - k}\\
	                   1      & 0         & 0         & \dots & 0 & 0 \\
	                   0      & 1         & 0         & \dots & 0 & 0\\
	                   0      & 0         & \ddots\\
	                   \vdots &           &           & \ddots\\
	                   0      & 0         & 0         & \dots & 1 & 0
	    \end{bmatrix}
\end{equation}

Dowód tej konstrukcji pozostawiamy czytelnikowi jako ćwiczenie.


\include{strassen}

\section{Model afinicznych drzew decyzyjnych}

Zdefiniujmy następujący problem (ang. element uniqness).
Mając daną tablicę \texttt{T[0..n-1]} liczb rzeczywistych, odpowiedzieć na pytanie czy istnieją w tablicy dwa elementy, które są sobie równe.
Pierwsze rozwiązanie jakie przychodzi wielu ludziom do głowy, to posortować tablicę \texttt{T} a następnie sprawdzić sąsiednie elementy.
Algorytm ten rozwiązuje nasz problem w czasie $\Theta(n \log n)$.
Pytanie - czy da się szybciej?
W niniejszym rozdziale udowodnimy, że w modelu afinicznych drzew decyzyjnych problemu nie da się rozwiązać lepiej.

W modelu afinicznych drzew decyzyjnych, w każdym zapytaniu możemy wybrać sobie $n+1$ liczb: $c$, $a_0$, $a_1$, $\dots$, $a_{n-1}$, a następnie zapytać czy
\[
 c + \sum_{i=0}^{n-1} a_i t_i \geq 0
\]
gdzie $t_i$ to elementy tablicy \texttt{T}.
Gdybyśmy użyli terminologii algebraicznej, to powiedzielibyśmy, że $t$ jest punktem w przestrzeni $\mathbb{R}^n$, lewa strona powyższej nierówności to przekształcenie afiniczne,
a zbiór wszystkich punktów z $\mathbb{R}^n$, które spełniają tą nierówność to przestrzeń afiniczna.
Jeśli na Algebrze nie wyrobiliście sobie jeszcze intuicji, to w $\mathbb{R}^2$ przestrzeń afiniczną otrzymujemy przez narysowanie dowolnej prostej i wzięcie wszystkich elementów z jednej ze stron.
Podobnie w $\mathbb{R}^3$ przestrzeń afiniczną otrzymujemy poprzez narysowanie dowolnej płaszczyzny, a następnie wzięcia wszystkich elementów z jednej ze stron.
W wyższych wymiarach wygląda to analogicznie.
\tizkboxwithcaption{tikz/uniqnesstree.tikz}{
Przykład afinicznego drzewa decyzyjnego.
W wierzchołkach wewnętrznych mamy zapytanie $(c^j, a_i^j)$.
W zależności od tego czy $c^k + \sum_{i=0}^{n-1} a^k_i t_i \geq 0$ czy też nie, idziemy odpowiednio w lewo lub w prawo.
Liście zawierają odpowiedź naszego algorytmu.
}

Algorytm używający tego typu porównań można zapisać za pomocą drzewa (rys. \ref{uniqness-tree}).
Zaczynamy z korzenia tego drzewa.
W każdym wierzchołku wewnętrznym zadajemy zapytanie.
W zależności od tego czy odpowiedż na pytanie była pozytywna czy negatywna, idziemy w drzewie w lewo lub w prawo.
Gdy dojdziemy do liścia w drzewie otrzymujemy nasze rozwiązanie (tak lub nie).
Takie drzewo będziemy nazywać afinicznym drzewem decyzyjnym.

Aby było nam łatwiej zdefiniować główny lemat naszego rozdziału, zdefiniujemy sobie dwa pojęcia.
\begin{definition}
 Mówimy, że punkt $t \in \mathbb{R}^n$ \textbf{osiąga} liść $l$ w afinicznym drzewie decyzyjnym, jeśli algorytm uruchomiony dla punktu $t$ dochodzi do liścia $l$.
\end{definition}
\begin{definition}
 Mówimy, że podzbiór $C \subseteq \mathbb{R}^n$ jest \textbf{zbiorem wypukłym}, jeśli dla dowolnych punktów $u, v \in C$ oraz dowolnej liczby rzeczywistej $0 \leq \alpha \leq 1$ punkt $\alpha \cdot u + (1 - \alpha) v$ także należy do $C$.
\end{definition}
Pierwsza z definicji pozwala nam mówić o elementach, które trafiają do tego samego liścia, a druga to sformalizowane pojęcie wypukłości znane z liceum.
Uzbrojeni w nowe definicje, możemy przejść do obiecanego lematu:

\begin{lemma}
 Zbiór punktów osiągających liść $l$ w afinicznym drzewie decyzyjnym, jest zbiorem wypukłym.
 \label{uniqness-lemma}
\end{lemma}

\begin{proof}
 Weźmy dowolne afiniczne drzewo decyzyjne i wybierzmy w nim dowolny liść $l$.
 Dobrze wiemy, że istnieje dokładnie jedna ścieżka prosta z korzenia do tego liścia.
 Weżmy dowolny wierzchołek wewnętrzny $w$ na tej ścieżce i dowolne punkty $u$ oraz $v$, które osiągają liść $l$.
 W końcu weźmy dowolną liczbę rzeczywistą $0 \leq \alpha \leq 1$.
 Załóżmy ponadto, że ścieżka z korzenia do liścia $l$ w wierzchołku $w$ skręca w lewo (zatem zapytanie zadane w wierzchołku $w$ punkty $u$ oraz $v$ otrzymały odpowiedź twierdzącą).
 Przypadek przeciwny jest analogiczny.
 Wiemy zatem, że
\[
 c^w + \sum_{i=0}^{n-1} a_i^w u_i \geq 0
\]
 oraz, że
\[
 c^w + \sum_{i=0}^{n-1} a_i^w v_i \geq 0
\]
 Ponieważ $\alpha \geq 0$ możemy przemnożyć pierwsze równanie przez $\alpha$:
\[
 \alpha c^w + \sum_{i=0}^{n-1} a_i^w \alpha u_i \geq 0
\]
 a ponieważ $1 - \alpha \geq 0$ możemy drugie równanie przemnożmyć przez $1 - \alpha$:
\[
 (1 - \alpha) c^w + \sum_{i=0}^{n-1} a_i^w (1 - \alpha) v_i \geq 0
\]
Teraz sumując oba równania otrzymujemy:
\[
 c^w + \sum_{i=0}^{n-1} a_i^w (\alpha u_i + (1 - \alpha) v_i) \geq 0
\]
Zatem punkt $\alpha \cdot u + (1 - \alpha) v$ również w wierzchołku $w$ skręci w tą samą stronę co punkty $u$ oraz $v$.
Ponieważ wybraliśmy dowolny wierzchołek $w$, to punkt $\alpha \cdot u + (1 - \alpha) v$ osiągnie liść $l$.
Stąd zbiór wszystkich punktów, które osiągają liść $l$ w afinicznym drzewie decyzyjnym, jest zbiorem wypukłym.
\end{proof}

Wyobraźmy sobie, że płaszczyznę $\mathbb{R}^2$ kładziemy półpłaszczyzny.
Wtedy przecięcie dowolnej liczby półpłaszczyzn jest zbiorem wypukłym.
Podobnie jeśli w przestrzeni $\mathbb{R}^3$ wyznaczymy sobie półprzestrzenie, to ich przecięcie będzie tworzyło zbiór wypukły.
Lemat \ref{uniqness-lemma} mówi, że tak samo się dzieje w każdej przestrzeni $\mathbb{R}^n$.

Ten lemat za chwilę okaże się dla nas kluczowy, gdyż za jego pomocą udowodnimy, że jeśli afiniczne drzewo decyzyjne poprawnie rozwiązuje problem element uniqness to musi posiadać conajmniej $n!$ liści.
Oznacza to, że wysokość takiego drzewa musi wynosić conajmniej $\O(n \log n)$.

\begin{lemma}
 \label{permutation-lemma}
 Niech $\{r_0, r_1, \ldots, r_{n-1}\}$ będzie $n$ elementowym zbiorem liczb rzeczywistych i niech $(a_0, a_1, \ldots, a_{n-1})$ oraz $(b_0, b_1, \ldots, b_{n-1})$ będą dwoma różnymi permutacjami liczb z tego zbioru.
 W każdym afinicznym drzewie decyzyjnym poprawnie rozwiązującym problem element uniqness, punkty $a$ oraz $b$ osiągają różne liście w drzewie.
\end{lemma}

\begin{proof}
 Dowód niewprost.
 Załóżmy, że $a$ oraz $b$ osiągają ten sam liść w drzewie.
 Liść ten musi odpowiadać przecząco na zadany problem, gdyż ani $a$ ani $b$ nie zawierają dwóch tych samych elementów.
 Ponieważ $a$ oraz $b$ składają się z tych samych liczb rzeczywistych i różnie się od siebie permutacją, to muszą istnieć takie indeksy $i$ oraz $j$, że $a_i > a_j$ oraz $b_i < b_j$.
 Weźmy następującą wartość $\alpha$:
 \[
  \alpha = \frac{b_j - b_i}{(a_i - a_j) + (b_j - b_i)}
 \]
 Wykonując proste przekształcenia arytmetyczne, możemy przekonać się, że $0 < \alpha < 1$.
 Oznacza to, na mocy lematu \ref{uniqness-lemma}, że punkt $\alpha a + (1 - \alpha)b$ również osiąga ten sam liść co punkty $a$ i $b$.
 Ponieważ jednak zachodzi:
 \[
  \alpha a_i + (1 - \alpha) b_i = \alpha a_j + (1 - \alpha) b_j
 \]
 (o czym można się przekonać wykonując proste przekształcenia arytmetyczne), odpowiedź algorytmu dla tego punktu powinna być twierdząca.
 Zatem afiniczne drzewo decyzyjne dla tego punktu zwraca złą odpowiedź.
 Sprzeczność z założeniem, że drzewo rozwiązywało problem poprawnie.
\end{proof}

Weźmy dowolny $n$ elementowy zbiór liczb rzeczywistych.
Na mocy Lematu \ref{permutation-lemma} każda permutacja tych liczb musi osiągać inny liść w afinicznym drzewie decyzyjnym poprawnie rozwiązującym problem element uniqness.
Oznacza to, że liczba liści w takim drzewie musi wynosić przynajmniej $n!$.
Zatem wysokość takiego drzewa musi wynosić conajmniej $O(n \log n)$.

\chapter{Under construction}

\include{zlozonosc}

\include{modelobliczen}

\include{kopiec}

\section{Algorytm rosyjskich wieśniaków}

\begin{algorithm}[h]
  \DontPrintSemicolon
  \SetAlgorithmName{Algorytm}{}
  
  \KwData{ $a$, $b$ - liczby naturalne }
  
  \KwResult{ $wynik = a \cdot b$ }
  
  $a' \leftarrow a$\;
  $b' \leftarrow b$\;
  $wynik \leftarrow 0$\;
  \While{\upshape $a' > 0$}
  {
    \If{\upshape $a' \textsf{ mod } 2 = 1$}
    {
      $wynik \leftarrow wynik + b'$\;
    }
    $a' \leftarrow a' \textsf{ div } 2$\;
    $b' \leftarrow b' \cdot 2$\;
  }
  
  \caption{Algorytm rosyjskich wieśniaków}
  \label{alg-wiesniakow}
\end{algorithm}

\comment{Jakiś wstęp, do czego służy algorytm, przykład działania, analogia między algorytmem a algorytmem szybkiego potęgowania.
Dowód poprawności jako jakieś twierdzenie / lemat.
Popracować nad składaniem latexa.
Stawiać entery po każdym zdaniu.}
Niech $a'_i$ (kolejno: $b'_i$, $wynik_i$) będzie wartością $a'$ ($b'$, $wynik$) w $i-tej$ iteracji pętli \texttt{while}. Udowodnimy następujący niezmiennik: $a'_i \cdot b'_i + wynik_i = a \cdot b$. Załóżmy, że niezmiennik zachodzi w $i-tej$ iteracji i sprawdźmy co dzieje się w $i+1$ iteracji. Rozważmy dwa przypadki.


\begin{itemize}
    \item $a'_i$ parzyste. Instrukcja \texttt{if} się nie wykona, w $i+1$ iteracji $wynik_i$ pozostanie niezmieniony, $a'_i$ zmniejszy się o połowę, a $b'_i$ zwiększy dwukrotnie. 

$a'_{i+1} \cdot b'_{i+1} + wynik_{i+1} = \frac{a'_i}{2} \cdot 2 b'_i + wynik_i = a_i \cdot b_i$


\item $a'_i$ nieparzyste:

$wynik_{i+1} \leftarrow wynik_i + b'_i$;  
$a'_{i+1} \leftarrow a'_i \textsf{ div } 2 = \frac{a'_i-1}{2}$;  
$b'_{i+1} \leftarrow b'_i \cdot 2$

Ostatecznie otrzymujemy:

$a'_{i+1} \cdot b'_{i+1} + wynik_{i+1} = \frac{a'_i-1}{2} \cdot 2 b'_i + wynik_i +b'_i = a'_i \cdot wynik_i + b'_i= a \cdot b$

\end{itemize}

Teraz wystarczy zauważyć, że tuż po wyjściu z pętli \texttt{while} wartość zmiennej $a'$ wynosi $0$. Podstawiając do niezmiennika okazuje się, że faktycznie algorytm rosyjskich wieśniaków liczy $a \cdot b$.

\paragraph{Złożoność}

Z każdą iteracją połowimy $a'$. Biorąc pod uwagę kryterium jednorodne pozostałe instrukcje w pętli nic nie kosztują. Stąd złożoność to $O(\log a)$.

W kryterium logarytmicznym musimy uwzględnić czas dominującej instrukcji: dodawania  $wynik \leftarrow wynik + b'$. W najgorszym przypadku zajmuje ono $O(\log ab)$. Zatem złożoność to $O(\log a \cdot \log ab)$.

\include{sortowanietopologiczne}

\include{sortowanie}

\include{mst}

\include{dijkstra}

\include{szeregowanie}

\include{dynamicznenadrzewach}

%% Dodatki

\begin{appendices}

\pagestyle{empty}
\input{porownanie.tex}

\end{appendices}

\end{document}