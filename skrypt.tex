\documentclass[10pt,b5paper]{book}
\usepackage{polski}
\usepackage{amsthm}
\usepackage[T1]{fontenc}
\usepackage[utf8]{inputenc}
\usepackage{geometry}
\usepackage{tikz}
\usepackage{amsmath}
\usepackage{amssymb}
\usepackage{array}
\usepackage[titletoc]{appendix}
\usepackage{scrextend}
\usepackage[ruled,commentsnumbered]{algorithm2e}
\usepackage{adjustbox}
\usepackage[hidelinks]{hyperref}

%% Biblioteki TiKZa

\usetikzlibrary{shapes.arrows}

%% Użyteczne komendy

\newtheorem{theorem}{Twierdzenie}
\newtheorem{definition}{Definicja}
\newtheorem{lemma}{Lemat}
\newtheorem{fact}{Fakt}
\newtheorem{observation}{Obserwacja}
\def\checkmark{\tikz\fill[scale=0.4](0,.35) -- (.25,0) -- (1,.7) -- (.25,.15) -- cycle;} 
\newcommand{\tizkboxwithcaption}[2]{\noindent\begin{figure}[!h]\centering\begin{adjustbox}{min width=0.75\textwidth, max width=1\textwidth}\input{#1}\end{adjustbox}\caption{#2}\end{figure}}

\newcommand{\comment}[1]{\marginpar{\tiny\raggedright #1}}

\renewcommand{\KwData}{\textbf{Input:}}
\renewcommand{\KwResult}{\textbf{Output:}}

%% Strona tytułowa

\usepackage[load-headings]{exsheets}
\DeclareInstance{exsheets-heading}{mylist}{default}{
  runin = true ,
  attach = {
    main[l,vc]number[l,vc](-3em,0pt) ;
    main[r,vc]points[l,vc](\linewidth+\marginparsep,0pt)
  }
}

\SetupExSheets{
  headings = mylist ,
  headings-format = \normalfont ,
  counter-format = se.qu ,
  counter-within = section
}

\usepackage{etoolbox}
\AtBeginEnvironment{question}{\addmargin[3em]{0em}}
\AtEndEnvironment{question}{\endaddmargin}

\newcommand*{\titleGM}{\begingroup
\hbox{
\hspace*{0.2\textwidth}
\hspace*{0.05\textwidth}
\parbox[b]{0.75\textwidth}{
{\noindent\Huge\bfseries Skrypt \\[0.2\baselineskip] z Algorytmów \\[0.2\baselineskip] i~struktur danych}\\[1\baselineskip]
{\large \textit{Zbiór mniej lub bardziej ciekawych algorytmów i~struktur danych, jakie bywały omawiane na wykładzie (albo i nie).}}\\[3\baselineskip]
{\Large \textsc{praca zbiorowa pod redakcją \\[0.1\baselineskip] Krzysztofa Piecucha}}

\vspace{0.5\textheight}
{\noindent Korzystać na własną odpowiedzialność.}\\[\baselineskip]
}}
\endgroup}

\definecolor{titlepagecolor}{cmyk}{0.9,1,.6,.40}

\newcommand\titlepagedecoration{%
\begin{tikzpicture}[remember picture,overlay,shorten >= -10pt]

\coordinate (aux1) at ([yshift=-15pt]current page.north west);
\coordinate (aux2) at ([yshift=-410pt]current page.north west);
\coordinate (aux3) at ([xshift=+4.5cm]current page.north west);
\coordinate (aux4) at ([yshift=-150pt]current page.north west);

\begin{scope}[titlepagecolor!60,line width=14pt,rounded corners=12pt]
\draw
  (aux1) -- coordinate (a)
  ++(-45:5) --
  ++(225:5.1) coordinate (b);
\draw[shorten <= -10pt]
  (aux3) --
  (a) --
  (aux1);
\draw[opacity=0.4,titlepagecolor,shorten <= -10pt]
  (b) --
  ++(-45:2.2) --
  ++(225:2.2);
\end{scope}
\draw[titlepagecolor,line width=9pt,rounded corners=8pt,shorten <= -10pt]
  (aux4) --
  ++(-45:1.2) --
  ++(225:1.2);
\begin{scope}[titlepagecolor!70,line width=6pt,rounded corners=6pt]
\draw[shorten <= -10pt]
  (aux2) --
  ++(-45:3) coordinate[pos=0.45] (c) --
  ++(225:3.1);
\draw
  (aux2) --
  (c) --
  ++(45:2.5) --
  ++(135:2.5) --
  ++(225:2.5) coordinate[pos=0.3] (d);   
\draw 
  (d) -- +(135:1);
\end{scope}
\end{tikzpicture}%
}

\begin{document}

\pagestyle{empty}
\titleGM
\titlepagedecoration

\tableofcontents

\chapter{Podstawy}

\include{zlozonosc}

\include{modelobliczen}

\chapter{Struktury danych}

\include{kopiec}

\chapter{Algorytmy}

\include{bitoniczne}

\section{Algorytm rosyjskich wieśniaków}

\begin{algorithm}[h]
  \DontPrintSemicolon
  \SetAlgorithmName{Algorytm}{}
  
  \KwData{ $a$, $b$ - liczby naturalne }
  
  \KwResult{ $wynik = a \cdot b$ }
  
  $a' \leftarrow a$\;
  $b' \leftarrow b$\;
  $wynik \leftarrow 0$\;
  \While{\upshape $a' > 0$}
  {
    \If{\upshape $a' \textsf{ mod } 2 = 1$}
    {
      $wynik \leftarrow wynik + b'$\;
    }
    $a' \leftarrow a' \textsf{ div } 2$\;
    $b' \leftarrow b' \cdot 2$\;
  }
  
  \caption{Algorytm rosyjskich wieśniaków}
  \label{alg-wiesniakow}
\end{algorithm}

\comment{Jakiś wstęp, do czego służy algorytm, przykład działania, analogia między algorytmem a algorytmem szybkiego potęgowania.
Dowód poprawności jako jakieś twierdzenie / lemat.
Popracować nad składaniem latexa.
Stawiać entery po każdym zdaniu.}
Niech $a'_i$ (kolejno: $b'_i$, $wynik_i$) będzie wartością $a'$ ($b'$, $wynik$) w $i-tej$ iteracji pętli \texttt{while}. Udowodnimy następujący niezmiennik: $a'_i \cdot b'_i + wynik_i = a \cdot b$. Załóżmy, że niezmiennik zachodzi w $i-tej$ iteracji i sprawdźmy co dzieje się w $i+1$ iteracji. Rozważmy dwa przypadki.


\begin{itemize}
    \item $a'_i$ parzyste. Instrukcja \texttt{if} się nie wykona, w $i+1$ iteracji $wynik_i$ pozostanie niezmieniony, $a'_i$ zmniejszy się o połowę, a $b'_i$ zwiększy dwukrotnie. 

$a'_{i+1} \cdot b'_{i+1} + wynik_{i+1} = \frac{a'_i}{2} \cdot 2 b'_i + wynik_i = a_i \cdot b_i$


\item $a'_i$ nieparzyste:

$wynik_{i+1} \leftarrow wynik_i + b'_i$;  
$a'_{i+1} \leftarrow a'_i \textsf{ div } 2 = \frac{a'_i-1}{2}$;  
$b'_{i+1} \leftarrow b'_i \cdot 2$

Ostatecznie otrzymujemy:

$a'_{i+1} \cdot b'_{i+1} + wynik_{i+1} = \frac{a'_i-1}{2} \cdot 2 b'_i + wynik_i +b'_i = a'_i \cdot wynik_i + b'_i= a \cdot b$

\end{itemize}

Teraz wystarczy zauważyć, że tuż po wyjściu z pętli \texttt{while} wartość zmiennej $a'$ wynosi $0$. Podstawiając do niezmiennika okazuje się, że faktycznie algorytm rosyjskich wieśniaków liczy $a \cdot b$.

\paragraph{Złożoność}

Z każdą iteracją połowimy $a'$. Biorąc pod uwagę kryterium jednorodne pozostałe instrukcje w pętli nic nie kosztują. Stąd złożoność to $O(\log a)$.

W kryterium logarytmicznym musimy uwzględnić czas dominującej instrukcji: dodawania  $wynik \leftarrow wynik + b'$. W najgorszym przypadku zajmuje ono $O(\log ab)$. Zatem złożoność to $O(\log a \cdot \log ab)$.

\section{Algorytm macierzowy wyznaczania liczb Fibonacciego}

W tym rozdziale opiszemy algorytm obliczania liczb Fibonacciego, który wykorzystuje szybkie 
potęgowanie\footnote{\url{https://en.wikipedia.org/wiki/Exponentiation_by_squaring}}. 
Algorytm działa w czasie $O(\log{n})$, co sprawia, że jest znacznie atrakcyjniejszy od algorytmu 
dynamicznego, który wymaga czasu $O(n)$.

Znajdźmy taką macierz $M$, która po wymnożeniu przez transponowany wektor wyrazów 
$F_{n}$ i $F_{n - 1}$ da nam wektor, w którym otrzymamy wyrazy $F_{n + 1}$ oraz $F_{n}$. 
Łatwo sprawdzić, że dla ciągu Fibonacciego taka macierz ma postać:


\begin{equation}
	M = \begin{bmatrix}1 & 1\\1 & 0\end{bmatrix}
\end{equation}
Bo:
\begin{equation}
\label{eq:fibonacci_m}
	M \times
	\begin{bmatrix}F_n \\ F_{n - 1}\end{bmatrix}
	= \begin{bmatrix}F_{n + 1} \\ F_{n}\end{bmatrix}
\end{equation}

Wynika to wprost z definicji mnożenia macierzy oraz definicji ciągu Fibonacciego.


\begin{observation}{Zauważmy, że możemy $M$ przemnożyć przez macierz otrzymaną w \ref{eq:fibonacci_m}. Otrzymamy wtedy macierz postaci:}
\begin{equation}
	M \times (M \times \begin{bmatrix}F_n \\ F_{n - 1}\end{bmatrix})
\end{equation}
\end{observation}

\begin{observation}{A gdy zrobimy to $n$ razy...}
\label{obs:mult-n-times}
\begin{equation}
	M \times (M \times (M \times ...\, (M \times \begin{bmatrix}F_n \\ F_{n - 1}\end{bmatrix})\,...\,))
\end{equation}
\end{observation}

\begin{fact}{Mnożenie macierzy jest łączne.}
\label{fact:mult-is-associative}
\end{fact}

Z Faktu \ref{fact:mult-is-associative}. i Obserwacji \ref{obs:mult-n-times}. mamy:

\begin{equation}
	M^n \times \begin{bmatrix}F_n \\ F_{n - 1}\end{bmatrix}
\end{equation}

Pokażemy, że powyższa macierz ma zastosowanie w obliczaniu n-tej liczby Fibonacciego.

\begin{lemma}
\begin{equation}
	M^{n} \times \begin{bmatrix}F_1 \\ F_0\end{bmatrix} = \begin{bmatrix}F_{n + 1} \\ F_{n}\end{bmatrix}
\end{equation}
\end{lemma}

\begin{proof}{Przez indukcję.}\\
Sprawdźmy dla $n = 1$. Mamy:
\begin{equation}
	\begin{bmatrix}1 & 1\\1 & 0\end{bmatrix}^1 \times \begin{bmatrix}1 \\ 0\end{bmatrix}
	= \begin{bmatrix}1 \\ 1\end{bmatrix} = \begin{bmatrix}F_2 \\ F_1\end{bmatrix}
\end{equation}

Rozważmy $n + 1$ zakładając poprawność dla $n$.

\begin{equation}
	\begin{bmatrix}1 & 1\\1 & 0\end{bmatrix}^{n + 1} \times \begin{bmatrix}1 \\ 0\end{bmatrix}
	= \begin{bmatrix}1 & 1\\1 & 0\end{bmatrix} \times \begin{bmatrix}1 & 1\\1 & 0\end{bmatrix}^{n} \times \begin{bmatrix}1 \\ 0\end{bmatrix}
	= \begin{bmatrix}1 & 1\\1 & 0\end{bmatrix} \times \begin{bmatrix}F_{n+1} \\ F_{n}\end{bmatrix}
	\stackrel{\ref{eq:fibonacci_m}}{=} \begin{bmatrix}F_{n + 2} \\ F_{n + 1}\end{bmatrix}
\end{equation}
\end{proof}

\begin{algorithm}[h]
	\DontPrintSemicolon
	\SetAlgorithmName{Algorytm}{}
	
	\KwData{ n }
	
	\KwResult{ $n+1$-sza liczba Fibonacciego }

	$M \leftarrow \begin{bmatrix}1 & 1\\1 & 0\end{bmatrix}$\;
	$M' \leftarrow \texttt{exp\_by\_squaring(M, n)}$\;
	
	\KwRet{$\texttt{pierwszy element wektora}\,(M' \times \begin{bmatrix}1 \\ 0\end{bmatrix})$}\;

	\caption{Procedura \texttt{get\_fibonacci}}
\end{algorithm}

Mimo że powyższy algorytm działa w czasie $O(\log{n})$, warto mieć na uwadze fakt, że liczby Fibonacciego 
rosną wykładniczo. W praktyce oznacza to pracę na liczbach przekraczających długość słowa maszynowego.

Zaprezentowaną metodę można uogólnić na dowolne ciągi, które zdefiniowane są przez liniową 
kombinację skończonej liczby poprzednich elementów. Wystarczy znaleźć odpowiednią macierz $M$; 
dla ciągów postaci $G_{n + 1} = a_n G_n + a_{n - 1} G_{n - 1} + ... + a_{n - k} G_{n - k}$ jest to:
\begin{equation}
	M = \begin{bmatrix}a_n    & a_{n - 1} & a_{n - 2} & \dots & a_{n - k} & a_{n - k}\\
	                   1      & 0         & 0         & \dots & 0 & 0 \\
	                   0      & 1         & 0         & \dots & 0 & 0\\
	                   0      & 0         & \ddots\\
	                   \vdots &           &           & \ddots\\
	                   0      & 0         & 0         & \dots & 1 & 0
	    \end{bmatrix}
\end{equation}

Dowód tej konstrukcji pozostawiamy czytelnikowi jako ćwiczenie.


\include{sortowanietopologiczne}

\include{sortowanie}

\include{mst}

\include{dijkstra}

\include{szeregowanie}

\include{dynamicznenadrzewach}

%% Dodatki

\begin{appendices}

\input{porownanie.tex}

\end{appendices}

\end{document}