\section{Algorytm Borůvki}
\sectionauthor{Michał Bronikowski}

\label{sec:boruvka}

\paragraph{} Algorytm Borůvki jest przykładem algorytmu zachłannego znajdującego minimalne drzewo rozpinające. Algorytm został opracowany przez 
Czeskiego matematyka Otakara Borůvke w 1926 roku. Miał pomóc zoptymalizować zasięg sieci elektrycznej w Czechach. 

\subsection{Działanie}
Niech $G = (V, E)$ będzie grafem ważonym, w którym wszystkie krawędzie mają różne wagi.
W pierwszym kroku algorytm Borůvki wybiera dla wszystkich wierzchołków $ v \in V$ najlżejszą krawędź $e \in E$,
która jest z nimi incydentna i dołącza  wierzchołki wraz z wybranymi krawędziami do lasu $F$. W następnym kroku 
tworzymy graf $G'$ będący kopią $G$, z tą różnicą, że wszystkie spójne składowe z $F$ zostały ze sobą ``połączone'' w pojedyńcze wierzchołki zwane      
superwierzchołkami. Powyższe czynności powtarzamy zamieniając $G'$ na $G$, dopóki nie otrzymamy jednej spójnej składowej, będzie to nasze $MST$.

\subsection{Pseudokod}

\begin{algorithm}[H]
  \DontPrintSemicolon
  \SetAlgorithmName{Algorytm}{}\\
  \KwData{ $\mathcal{G}$ - Graf, w którym szukamy MST }\\
  \KwResult{ $\mathcal{T}$, takie że $\mathcal{T}$ jest $MST$ $\mathcal{G}$ }\\
$\mathcal{T} \leftarrow \emptyset$  \\
$\mathcal{F} \leftarrow \emptyset$  \\
  \While{$|\mathcal{G}(V)| > 1$}
  {
     Do $\mathcal{F}$ dołącz wszystkie wierzchołki z $\mathcal{G}$ wraz z incedentnymi krwędziami o najniższej wadze\\
     W $\mathcal{F}$ wszystkie spójne składowe zamień na pojedyńcze wierzchołki\\
     $\mathcal{G'} \leftarrow \mathcal{G} \setminus F$ \\
     $\mathcal{G} \leftarrow \mathcal{G'}$
  }
  \caption{Algorytm Borůvki}
  \label{alg-boruvka}
\end{algorithm}

\subsection{Dowód poprawnośći}
W każdym kroku algorytmu budujemy rozwiązanie będące $MST$, poprzez dodanie do niego spójnych składowych z $F$ w postaci jednego wierzchołka, a z
$Cut\ Property$ wiemy, że te wierzchołki należą do $MST$

\subsection{Złożoność czasowa}
W pojedyńczym kroku algorytmu wykonujemy maksymalnie $m$ operacji, gdzie $m$ jest ilością krawędzi w $G$.
W każdym kroku algorytmu ilość wierzchołków zmniejsza się przynbajmniej o połowę.
W związku z tym złożoność algorymu Borůvki wynosi $O(m\log{}n)$, gdzie $n$ oznacza ilość wierzchołków w $G$.
