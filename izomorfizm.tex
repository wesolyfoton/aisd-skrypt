\section{Izomorfizm drzew}
\sectionauthor{Dominik Bruliński}

Nieformalnie możemy powiedzieć, że dwa grafy są izomorficz jeżeli jeden może być przekształcony w drugi za pomocą zmiany nazwy wierzchołków.

Zanim przejdziemy do sprawdzania czy dwa drzewa są izomorficzne spójrzmy na kilka definicji.

\subsection{Definicje}

\begin{definition}

    \textbf{Izomorfizm grafów} $G_1(V_1, E_1)$ i $G_2(V_2, E_2)$ jest to bijekcja (mapowanie 1:1) między zbiorami wierzchołków $\varphi
    : V_1 \rightarrow V_2$, taka, że:
$$ \forall u, v \in V_1 (u, v) \in E_1 \iff (\varphi(u),\varphi(v)) \in E_2 $$


\end{definition}

\begin{definition}

    \textbf{Ukorzenione drzewo} $(V, E, r)$ to drzewo $(V, E)$ z wybranym korzeniem $r \in V$.

\end{definition}

\begin{definition}
    \textbf{Izomorfizm ukorzenionych drzew} $T_1(V_1, E_1, r_1)$ i $T_2(V_2, E_2, r_2)$ to bijekcja między zbiorami wierzchołków
    $\varphi : V_1 \rightarrow V_2$, taka, że:
    $$\forall u, v \in V_1 (u, v) \in E_1 \iff (\varphi(u),\varphi(v)) \in E_2 $$
    oraz
    $$\varphi(r_1) = r_2$$
\end{definition}

\begin{definition}
    \textbf{Centroidy} to wierzchołki leżące po środku drzewa - a dokładniej takie, które dzielą je na poddrzewa o rozmiarze nie większym niż połowa całego drzewa. Innymi słowy centroid to wierzchołek v taki, że najdłuższa ścieżka z v do liścia jest minimalna z pośród wszystkich wierzchołków (połowa najdłuższej ścieżki).

    \begin{algorithm}[H]
        \DontPrintSemicolon
        \SetAlgorithmName{Znajdowanie centroidu}{}\\
        \KwData{ $\mathcal{T}$ - Drzewo, w którym szukamy centoridu }\\
        \KwResult{ $\mathcal{v}$, - wierzchołek będący centroidem  }\\
        1. Wybieramy losowy wierzchołek $r$ \\
        2. Znajdujemy wierzchołek $v_1$ - najbardziej oddalony od $r$ \\
        3. Znajdujemy wierzchołek $v_2$ - najbardziej oddalony od $v_1$ \\
        4. Centrum znajduje się na środku najdłuższej ścieżki między $v_1$ i $v_2$ \\
        \caption{Znajdowanie cetroidu}
    \end{algorithm}

Złożoność powyższego algorytmu to $O(n)$

\end{definition}

\begin{lemma}
Jeżeli istnieje algorytm $O(n)$ dla izomorfizmów ukorzenionych drzew, to istnieje algorytm $O(n)$ dla nieukorzenionych drzew.
\end{lemma}

\begin{proof}

Niech $A$ będzie algorytmem dla ukorzenionych drzew działającym w czasie $O(n)$, a $T_1$ i $T_2$ będą nieukorzenionymi drzewami.

Znajdźmy centra tych drzew. Mamy trzy przypadki:
\begin{itemize}
\item obydwa drzewa mają tylko jedno centrum (odpowiednio $c_1$ i $c_2$) rozwiązaniem wtedy będzie $A(T_1, c_1, T_2, c_2)$
\item każde drzewo ma dokładnie dwa centra (odpowiednio $c_1$, $c_1\prime$ oraz $c_2$, $c_2\prime$) rozwiązanie to $A(T_1, c_1, T_2, c_2)$ bądź $A(T1, c_1\prime,T_2, c_2\prime)$
\item gdy drzewo ma inną liczbę centrów nie mamy rozwiązania

\end{itemize}

\end{proof}

\subsection{Algorytm}

Poniższy algorytm sprawdza czy dwa ukorzenione drzewa $T_1$ i $T_2$ o $n$ wierzchołkach są izomorficzne w czasie $O(n)$
Algorytm przypisuje dodatkowe wartości wierzchołkom obydwu drzew zaczynając od liści i idąc w górę w taki sposób, że drzewa są izomorficzne tylko jeśli w korzeniach mają tą samą dodatkową wartość.

\begin{algorithm}[H]
    \DontPrintSemicolon
    \SetAlgorithmName{Izomorfizm drzew}{}\\
    \KwData{ $\mathcal{T_1}, \mathcal{T_2}$ - Ukorzenione drzewa}\\
    \KwResult{ $true$ jeśli drzewa są izomorficzne $false$ w.p.p, - wierzchołek będący centroidem  }\\
    wszystkim wierzchołkom z $T_1$ i $T_2$ przypisujemy wysokość w drzewie oraz pustą krotkę\\
    do wszystkich liści dodajemy dodatkową wartość $0$\\
    tworzymy listę $L_1$ zawierającą liście z drzewa $T_1$ na poziomie 0 \\
    tworzymy listę $L_2$ zawierającą liście z drzewa $T_2$ na poziomie 0 \\
    \For {wszystkie poziomy zaczynając od $1$} {
        \For {wszystkie wierzchołki $v$ na liście $L_1$} {
            dodaj do krotki rodzica dodatkową wartość dla $v$
        }
        niech $S_1$ będzie zbiorem krotek utworzonych dla nie-liści z $T_1$ na poziomie $i$ \\
        niech $S_2$ będzie odpowiadającym zbiorem dla $T_2$\\
        sortujemy $S_1$ i $S_2$ // bucket sort w $O(n)$ \\
        \uIf {$S_1 \ne S_2$} {
            return False // $T_1 i T_2$ nie są izomorficzne \\
        } \Else {
            czyścimy $L_1$ \\
            \For {$k$ od 1 do liczba wszystkich krotek z $S_1$} {
                \For {wszystkie wierzchołki $v$ z $T_1$ na poziomie $i$ reprezentowane przez $k$-tą krotkę w $S_1$} {
                    przypisz dodatkową wartość $k$ do wierzchołka $v$ \\
                    dodaj wierzchołek $v$ do $L_1$ \\
                }
            }
            dodaj na początek $L_1$ liście z $T_1$ na poziomie $i$ \\
            wykonaj powyższe kroki dla $L_2$ i $T_2$ \\
        }
    }
    \uIf {korzenie $T_1$ i $T_2$ mają tą samą dodatkową} {
        return True // $T_1$ i $T_2$ są izomorficzne
    }
    \Else {
        return False // $T_1$ i $T_2$ \textbf{nie} są izomorficzne
    }
    \caption{Izomorfizm drzew}
\end{algorithm}

Uwagi:
\begin{itemize}
\item Jeżeli będziemy rozpatrywali najpierw wszystkie liści (niezależnie od poziomu) nie będzie potrzebne sortowanie
\item krotki można hashować co pozwala je łatwiej dopasowywać
\end{itemize}

Powyższy algorytm nazywany jest AHU od nazwisk autorów\footnote{A. V. Aho, J. E. Hopcroft, and J. D. Ullman. The Design and Analysis of Computer Algorithms. Addison-Wesley, 1974}.

