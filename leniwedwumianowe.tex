\section{Kopce dwumianowe w wersji leniwej}
\sectionauthor{Marcin Bartkowiak}

\label{sec:leniwedwumianowe}

Kopce dwumianowe w wersji leniwej różnią się strukturalnie od wersji gorliwej, tym, że w danym momencie możemy mieć więcej niż jeden kopiec $B_k$ na liście.
\subsection{Różnice w implementacji}

\subsubsection{insert}
Stwórz drzewo składające się wyłącznie z danego elementu a następnie wywołaj \texttt{meld} z kopcem właściwym.
\subsubsection{meld}
Operacja \texttt{meld} polega na połączeniu list drzew dwóch kopców.
\subsubsection{extract-min}
W wersji leniwej, podobnie jak w Kopcach Fibonacciego
to tutaj będziemy wykonywać całą pracę związaną z utrzymaniem struktury kopców.

Idea algorytmu:
\begin{enumerate}
 \item Usuń min z listy wierzchołków, a następnie dodaj do listy wierzchołków jego dzieci.
 \item Stwórz pustą tablicę $B$ wielkości największemu stopniowi drzewa niezbędnego, w kopcu o poprawnej strukturze trzymającym wszystkie elementy($\lceil \log n \rceil$)
 \item Iterując po każdym drzewie w kopcu sprawdź, czy to nie minimum(i ustaw wskaźnik minimum jeśli jest),
      a następnie sprawdź w tablicy $B$ jest element o indeksie jego wielkości.
      Jeśli nie ma wstaw go do tablicy. Jeśli istnieje połącz dane drzewa i rekurencyjnie wstaw nowe drzewo do tablicy.
\end{enumerate}

\subsubsection{Pozostałe operacje}
Pozostałe operacje implementujemy identycznie jak w gorliwych kopcach dwumianowych.

\subsection{Analiza złożoności}
Zdefiniujmy funkcję potencjału $\Phi = \#drzew~w~kopcu$
\subsubsection{meld}
\texttt{Meld} w oczywisty sposób nie zmienia sumy potencjałów kopców, jedynym kosztem będzie przepięcie wskaźników, więc złożoność tej funkcji to $\Theta(1)$
\subsubsection{insert}
Dodając drzewo zwiększamy potencjał o jeden.
Następnie będziemy musieli wykonać \texttt{meld}, które kosztuje jedną operacje.
\[
  \Delta(\Phi) + 1 = 1 + 1 = 2 \in \Theta(1)
\]
\subsubsection{extract-min}
Na początku będziemy musieli wstawić wszystkie dzieci od minimalnego elementu; zajmie to $O(\log n)$.

Oznaczmy $T$ jako wszystkich drzew po tej operacji.

Dominującym kosztem rzeczywistym łączenia będzie iteracja po wszystkich drzewach(w czasie $\Theta(T)$).

Niech $\Delta(\Phi)$ oznacza różnicę potencjałów między kopcem po dodaniu dzieci minimalnego elementu, a kopcem po złączenie drzew tego samego stopnia.
Koszt zamortyzowany wyrażać się więc będzie wzorem.
\[
  \Delta(\Phi) + O(\log n) + \Theta(T) = O(\log(n)) - T + O(\log(n)) + \Theta(T) = O(\log(n))
\]

